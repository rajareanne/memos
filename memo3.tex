\documentclass[11pt]{article}
\newcommand{\thetitle}{HYPERION Memo \#1: The Relationship Between System 
Temperature and Sky Beam Coverage}
\newcommand{\theauthor}{Kara Kundert}
\newcommand{\theauthorsemail}{kkundert@berkeley.edu}
\newcommand{\thedate}{June 23, 2016}
% the following controls some aspects of how the text is displayed on the page
\setlength{\textwidth}{6.5in}
\setlength{\textheight}{8.25in}
\setlength{\oddsidemargin}{0in}

% set up the page headers and footers
\usepackage{fancyhdr}
    \pagestyle{fancy}
    \lhead{\sffamily\slshape\small\thetitle}
    \rhead{\sffamily\small\theauthor}
    \cfoot{\sffamily\slshape\small\thepage}

% support display of graphics
\usepackage{graphicx}

% the following control some aspects of how paragraphs are displayed
\parindent=0pt
\parskip=2ex

% import library of technical symbols
\usepackage{amsmath,amssymb,latexsym}

% import bibliography tools
\usepackage{natbib}
\citestyle{aa}

\begin{document}
% print the title in san-serif font, in bold, in huge characters
\title{
    \sffamily\bfseries\huge
    \thetitle \\
}
% print the author in san-serif font
\author{
    \sffamily\theauthor \\
    \sffamily\theauthorsemail
}
\date{\thedate}
\maketitle
\sloppy

\section{Introduction}


\section{Background and Theory}

\begin{equation}
    \label{eq:sys-temp}
    T_{sys} = T_{sky} * 10^{dB/10} + T_{rx} * (1 - 10^{dB/10})
\end{equation}

\secion{Method}


\section{Data and Analysis}

\begin{figure}
    \begin{center}
    \includegraphics[width=\linewidth]{/home/kara/capo/kmk/scripts/sysTemp.png}
    \end{center}
    \caption{
        Pictured above is the relationship between the fractional sky coverage 
        as determined by the beam-limiting baffles and the overall system 
        temperature. It is readily apparent that below $10\%$ sky coverage, the 
        system temperature is completely dominated by the presence of the 300 K 
        baffles.
    }
    \label{fig:sys-temp}
\end{figure}

\section{Conclusions}


\bibliography{hyperion}{}
\bibliographystyle{apj}

\end{document}





